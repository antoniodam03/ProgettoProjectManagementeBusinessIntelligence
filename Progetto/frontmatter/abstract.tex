%\selectlanguage{italian}
\begin{abstract}

Negli ultimi anni la Data Analytics e la Business Intelligence hanno assunto un ruolo sempre più centrale nei processi decisionali in ambito strategico, in particolare nei settori della Difesa e della Sicurezza. La capacità di analizzare grandi volumi di dati relativi a eventi di violenza politica e terrorismo consente di individuare pattern ricorrenti, monitorare l’evoluzione dei gruppi armati e valutare il livello di instabilità nelle diverse aree geografiche.

In questa tesi sono stati analizzati dati storici relativi alla minaccia terroristica globale. In particolare, è stata realizzata una fase di ETL (Extract, Transform and Load) finalizzata alla pulizia, integrazione e strutturazione del dataset, seguita da un’analisi descrittiva e dalla costruzione di dashboard interattive per la visualizzazione dei principali indicatori. L’obiettivo è fornire uno strumento di supporto alle decisioni capace di trasformare dati complessi in informazioni strategiche utili per il decision-making in ambito Difesa e Sicurezza.
\\[1cm]
\textbf{Keyword}: Data Analytics, Business Intelligence, Risk Management, Extract Trasform and Load, Qlik, Tableau, Power BI
\end{abstract} 

\selectlanguage{italian}