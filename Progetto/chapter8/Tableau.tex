\chapter{Tableau} \label{cap:tableau}

\begin{preamble}
{\em
    \textcolor{red}{\textbf{Da scrivere}}}
\end{preamble}

\section{Introduzione}

Il presente capitolo illustra la fase di \textbf{Data Visualization} e \textbf{Analisi Predittiva}, realizzata mediante l'utilizzo del software Tableau. La scelta di questo strumento è stata dettata dalla necessità di gestire un dataset complesso come il \textit{Global Terrorism Database} (GTD), caratterizzato da una profonda serie storica (1970--2017) e da una natura multivariata (geografica, temporale e categorica).

L'obiettivo principale del progetto è trasformare i dati grezzi in informazioni strategiche (\textit{Actionable Insights}). Non ci si è limitati a una rappresentazione descrittiva degli eventi passati, ma si è sfruttato il motore analitico di Tableau per implementare:

\begin{itemize}
    \item \textbf{Modelli di Forecasting:} basati su tecniche di smussamento esponenziale per proiettare i trend futuri.
    \item \textbf{Clustering:} per individuare pattern nascosti e isolare anomalie all'interno dei dati.
\end{itemize}

L'analisi mira dunque a isolare le dinamiche della minaccia nel futuro immediato e a medio termine, fornendo una visione d'insieme chiara e scientificamente fondata.


\section{Metodologia di Analisi}
L'approccio analitico adottato segue un flusso logico strutturato in tre fasi distinte, necessarie per rispondere alle \textit{Business Questions} definite in fase di progettazione:

\begin{enumerate}
    \item \textbf{Analisi Descrittiva (Cosa è successo?):} Esplorazione dei trend storici per comprendere l'evoluzione del fenomeno, la distribuzione geografica degli attacchi e l'impatto in termini di vittime ed economia.
    \item \textbf{Analisi Diagnostica (Perché è successo?):} Indagine sulle correlazioni tra variabili (es. relazione tra numero di feriti e numero di morti) e sulla segmentazione degli eventi tramite tecniche di clustering per isolare gli attacchi ad alta intensità.
    \item \textbf{Analisi Predittiva (Cosa potrebbe succedere?):} Applicazione di algoritmi predittivi per stimare i volumi futuri degli attacchi, l'evoluzione della letalità e i target a rischio, fornendo supporto ai processi di pianificazione strategica e di sicurezza.
\end{enumerate}

\section{Strategia di Visualizzazione}
Per garantire la massima leggibilità e interattività, le dashboard sono state progettate seguendo le \textit{best practices} della Data Visualization. Ogni visualizzazione è stata costruita per rispondere a una specifica domanda chiave, utilizzando filtri dinamici (per Regione, Anno o Tipo di Attacco) che permettono all'utente di esplorare i dati a diversi livelli di granularità.

Sono state impiegate diverse tipologie di grafici avanzati, tra cui:
\begin{itemize}
    \item \textbf{Dual Axis Chart:} per il confronto di metriche con scale differenti.
    \item \textbf{Scatter Plot:} per l'analisi di dispersione e correlazione.
    \item \textbf{Highlight Tables:} per la visualizzazione immediata delle densità di rischio tramite gradienti di colore.
    \item \textcolor{red}{\textbf{Altro?}}.

\end{itemize}

\section{Gestione della Qualità dei Dati}
Una fase critica antecedente alla visualizzazione ha riguardato la bonifica dei campi relativi alle vittime: \texttt{nkill} (numero di morti) e \texttt{nwound} (numero di feriti). L'analisi preliminare nel Pannello Dati di Tableau ha evidenziato due problematiche strutturali:

\begin{enumerate}
    \item \textbf{Errata Tipizzazione (Data Type Mismatch):} In fase di importazione, Tableau aveva interpretato queste colonne come stringhe. È stato necessario forzare il \textit{Type Casting} in \textbf{Numero Intero} per abilitare le aggregazioni matematiche (somme, medie).
    \item \textbf{Gestione dei Valori Nulli (Missing Values):} Numerosi record presentavano valori vuoti. È stata applicata la funzione \texttt{ZN()} (\textit{Zero Null}):
    \begin{quote}
        \texttt{nkill\_fix = ZN([nkill])} $\rightarrow$ Se il valore è nullo, viene convertito in 0.
    \end{quote}
\end{enumerate}

\textcolor{red}{\textbf{è stato fatto altro ? }}

\section{Panoramica delle Dashboard Realizzate}
Il progetto si articola in 7 Dashboard, organizzate in un percorso narrativo che parte dall'analisi storica globale per arrivare a previsioni specifiche su target e regioni:

\begin{itemize}
    \item \textbf{Dashboard 1:} Analisi Temporale degli Attacchi e delle Vittime (Trend Storici).
    \item \textbf{Dashboard 2:} Analisi e Previsione della Letalità per Regione.
    \item \textbf{Dashboard 3:} Previsione dei Bersagli (Target) per Regione.
    \item \textbf{Dashboard 4:} Analisi degli Esiti dell’Attacco (Successo vs Fallimento).
    \item \textbf{Dashboard z:} Previsione del Volume degli Attacchi.
    \item \textbf{Dashboard y:} Previsione dell’Impatto Economico per Tipo di Attacco.
    \item \textbf{Dashboard x:} Previsione dell’Intensità (Vittime) per Tipo di Attacco e Arma.

\end{itemize}


\subsection{Dashboard 1: Analisi Temporale degli Attacchi e delle Vittime}

\subsubsection*{Obiettivo dell'Analisi}
Questa dashboard rappresenta il punto di ingresso all'analisi storica, mettendo a confronto la \textbf{Frequenza} del fenomeno (Conteggio degli Attacchi -- Linea Blu) con la sua \textbf{Letalità Totale} (Somma delle Vittime/\texttt{nkill} -- Linea Rossa). L'obiettivo è verificare la correlazione tra volume di fuoco e impatto umano, rispondendo al quesito: \textit{"L'aumento del numero di attacchi si traduce sempre in un aumento proporzionale delle vittime, o esistono periodi in cui il terrorismo diventa intrinsecamente più letale?"}.

\subsubsection*{Configurazione della Visualizzazione}
È stato utilizzato un grafico \textbf{Dual Axis} (Doppio Asse) con scale indipendenti:
\begin{itemize}
    \item \textbf{Asse Sinistro (Attacchi):} Scala adattata al volume degli eventi.
    \item \textbf{Asse Destro (Vittime):} Scala adattata al numero dei deceduti.
\end{itemize}
Questa scelta tecnica è cruciale: sganciando la sincronizzazione degli assi, è possibile sovrapporre visivamente i trend per confrontare la "forma" delle curve, identificando immediatamente i momenti in cui la mortalità cresce a un ritmo diverso rispetto alla frequenza degli attacchi.

\textcolor{red}{\textbf{ Foto }}

\subsubsection*{Analisi del Trend (Confronto Attacchi vs Vittime)}
Osservando il grafico, emergono due fasi distinte:
\begin{enumerate}
    \item \textbf{Parallelismo (1970 -- 2000):} Le due linee si muovono quasi all'unisono. Quando gli attacchi salgono, le vittime salgono in proporzione. Questo indica una letalità media costante e una "capacità distruttiva" standardizzata degli attacchi.
    \item \textbf{Divergenza e Accelerazione (Post-2010):} È la fase più critica. Sebbene entrambe le curve crescano, la linea delle \textbf{Vittime} mostra un'accelerazione verticale più marcata rispetto alla linea degli \textbf{Attacchi}. Questo fenomeno suggerisce un cambiamento strutturale nel \textit{modus operandi}: negli ultimi anni del dataset, il terrorismo è diventato più efficiente nel causare vittime (attacchi di massa, uso di esplosivi più potenti), rompendo la correlazione lineare del passato.
\end{enumerate}

\subsubsection*{Validazione Statistica dei Modelli}
Per confermare matematicamente l'osservazione visiva, sono state applicate \textbf{Linee di Tendenza Polinomiali di 3° grado (Cubiche)}. I parametri statistici estratti confermano la robustezza dell'analisi:
\begin{itemize}
    \item \textbf{Significatività (P-Value):} Per entrambe le curve, il \textit{P-Value} è $< 0,0001$. Essendo ben al di sotto della soglia di $0,05$, possiamo affermare con certezza statistica che la crescita osservata non è casuale ma descrive un trend reale.
    \item \textbf{Confronto dei Coefficienti:}
    \begin{itemize}
        \item La curva delle Vittime (\texttt{nkill}) ha un coefficiente cubico ($x^3$) di $3,92 \cdot 10^{-8}$.
        \item La curva degli Attacchi ha un coefficiente cubico ($x^3$) di $1,76 \cdot 10^{-8}$.
    \end{itemize}
    \textit{Interpretazione:} Il coefficiente delle vittime è più del doppio di quello degli attacchi. Matematicamente, questo conferma che la curva della mortalità è più "ripida" e cresce più velocemente.
\end{itemize}

\textcolor{red}{\textbf{foto dati }}


\subsubsection*{Conclusioni}
L'analisi dimostra che il rapporto tra attacchi e vittime non è $1:1$. L'impennata della linea rossa e i coefficienti del modello polinomiale rivelano un aggravamento della minaccia nell'ultimo decennio analizzato: il terrorismo moderno non colpisce solo più spesso, ma colpisce "più forte".

\begin{center}
    \fbox{
        \parbox{0.9\textwidth}{
            \textbf{Nota Metodologica: Il "Missing Data" del 1993}\\
            I dati relativi al 1993 sono andati persi storicamente a causa di un problema durante il trasferimento degli archivi cartacei della \textit{Pinkerton Global Intelligence Service}. In fase di \textit{Data Cleaning}, si è scelto di mantenere il valore nullo piuttosto che interpolare artificialmente i dati, per preservare l'integrità scientifica. Tableau gestisce correttamente questa mancanza interrompendo la linea, segnalando visivamente l'assenza di informazioni per quell'anno specifico.
        }
    }
\end{center}

\subsection{Dashboard 2: Analisi e Previsione della Letalità per Regione}

\subsubsection*{Obiettivo dell'Analisi}
Questa dashboard indaga la relazione tra l'intensità dell'attacco (misurata in numero di feriti, \texttt{nwound}) e la sua mortalità (numero di morti, \texttt{nkill}). L'obiettivo è duplice: verificare la correlazione tra le due variabili e prevedere l'evoluzione della letalità media, permettendo confronti geografici mirati.

\textcolor{red}{\textbf{Foto}}

\begin{enumerate}
    \item \textbf{Analisi di Correlazione e Clustering (Scatter Plot):} La parte superiore ospita un grafico a dispersione dove ogni punto rappresenta un singolo evento.
    \begin{itemize}
        \item \textbf{Clustering K-Means:} Tramite l'algoritmo \textit{K-Means} di Tableau, gli attacchi sono stati raggruppati statisticamente per gravità.
        \item \textbf{Legge di Potenza:} La densa "nuvola" colorata vicino all'origine rappresenta il \textit{Cluster a Bassa-Alta Intensità}. I punti isolati (\textit{Outliers}) rappresentano gli "Eventi Critici", rari ma devastanti.
    \end{itemize}

    \item \textbf{Trend Storico e Previsione (Forecast):} La parte inferiore utilizza il motore di \textit{Exponential Smoothing} di Tableau. L'area ombreggiata attorno alla linea di previsione rappresenta l'intervallo di confidenza al 95\%. Anche in questo grafico è visibile l'interruzione dei dati corrispondente all'anno 1993.

    \item \textbf{Analisi Geografica (Filtro Dinamico):} L'implementazione di un filtro per regione permette di trasformare una media mondiale generica in uno strumento di \textit{Risk Assessment} specifico per area.
\end{enumerate}

\subsubsection{Caso Studio Comparativo: South Asia vs North America}
Per validare il modello, è stato eseguito un confronto diretto tra \textit{South Asia} (area ad alta instabilità) e \textit{North America} (area a stabilità relativa).

\textcolor{red}{\textbf{Foto caso studio}}


\paragraph{Analisi Qualitativa}
\begin{itemize}
    \item \textbf{South Asia:} La nuvola di punti nello \textit{Scatter Plot} è estremamente densa, confermando che gli eventi ad alta intensità non sono eccezioni ma ricorrenze statistiche.
    \item \textbf{North America:} Il grafico appare quasi "vuoto". La quasi totalità degli eventi si concentra nell'origine; i casi ad alta intensità sono rari (es. 11 settembre).
\end{itemize}

\paragraph{Analisi Predittiva e Business Insights}
L'analisi del \textit{Tasso di Letalità Previsto} (rapporto tra morti e feriti stimati) ha rivelato risultati inaspettati:

\begin{table}[h]
    \centering
    \begin{tabular}{lcl}
        \hline
        \textbf{Regione} & \textbf{Tasso di Letalità} & \textbf{Caratteristica della Minaccia} \\ \hline
        South Asia & 0,766 & Alta frequenza, letalità "diluita" (molti feriti) \\
        North America & 0,990 & Bassa frequenza, alta efficienza letale \\ \hline
    \end{tabular}
    \caption{Confronto del tasso di letalità previsto tra South Asia e North America.}
    \label{tab:confronto_letalita}
\end{table}

\begin{itemize}
    \item \textbf{South Asia (0,766):} Ogni 100 feriti si registrano circa 76 decessi. La tipologia di attacchi (spesso esplosivi in aree aperte) genera un numero massiccio di feriti. La minaccia è definita dal \textbf{volume} degli eventi.
    \item \textbf{North America (0,990):} Il rapporto è quasi $1:1$. Sebbene gli attacchi siano rari, mostrano una "efficienza letale" superiore, suggerendo l'uso di armi specifiche (sparatorie di massa) con esiti più frequentemente fatali.
\end{itemize}

\subsubsection*{Conclusioni Strategiche}
Il confronto evidenzia due profili di rischio opposti: nel \textbf{Sud Asia}, le strutture di emergenza devono focalizzarsi sulla gestione dei grandi flussi di feriti; in \textbf{Nord America}, la prevenzione deve concentrarsi sull'anticipazione dell'evento raro, data l'altissima probabilità di decesso in caso di attacco.


\subsubsection*{Validazione Statistica del Modello Predittivo}

\paragraph{Motore di Forecasting e Parametri di Base}
Per generare la proiezione futura della letalità, Tableau ha utilizzato un algoritmo di \textbf{Livellamento Esponenziale} (\textit{Exponential Smoothing}). Il modello selezionato automaticamente è di tipo \textbf{Aggiuntivo} per il Livello, senza Trend e senza Stagionalità (Modello: \textit{Aggiuntivo, Nessuno, Nessuno}).

L'algoritmo ha calcolato i seguenti valori per la previsione:
\begin{itemize}
    \item \textbf{Rapporto di Letalità Iniziale previsto:} $0,864$
    \item \textbf{Intervallo di confidenza (95\%):} $\pm 0,701$
\end{itemize}

\paragraph{Analisi delle Metriche di Qualità}
Il software ha classificato la qualità della previsione come \textit{"Scarsa"}. Tale valutazione non indica un errore di modellazione, ma riflette l'estrema volatilità del fenomeno. Le metriche specifiche chiariscono il quadro:

\begin{itemize}
    \item \textbf{RMSE ($0,358$) e MAE ($0,245$):} Misurano lo scostamento tra valori previsti e reali. Essendo contenuti, indicano che la linea di base è centrata correttamente sulla media storica.
    \item \textbf{MASE ($0,91$):} Essendo inferiore a $1$, certifica che il modello ha prestazioni superiori rispetto a un modello \textit{Naive}. Il sistema estrae dunque un pattern statistico valido.
    \item \textbf{MAPE ($27,8\%$):} L'errore percentuale spiega il rating di Tableau; esiste un margine d'incertezza del 30\% circa sulle previsioni.
\end{itemize}



\paragraph{Interpretazione Strategica del Modello}
Dal punto di vista dell'analisi di \textit{intelligence}, una qualità statistica definita "scarsa" a causa dell'alta varianza è un'informazione preziosa. L'ampiezza dell'intervallo di confidenza ($\pm 0,701$) dimostra matematicamente che il terrorismo non segue un trend lineare.

Gli eventi estremi possono alterare le medie in qualsiasi momento. Il modello suggerisce che non si deve basare la pianificazione solo sulla media attesa ($0,86$), ma si devono preparare le risorse di emergenza per l'estremo superiore dell'intervallo di confidenza, ovvero per potenziali picchi di letalità improvvisi.

\textcolor{red}{\textbf{figura dati previsione}}


\subsection{Dashboard 3: Previsione dei Bersagli (Target) per Regione}

\subsubsection*{Obiettivo dell'Analisi (Vittimologia)}
Mentre le dashboard precedenti si sono concentrate sui volumi globali e sull'impatto in termini di vite umane (letalità), questa visualizzazione sposta il focus sulla \textbf{Vittimologia}. L'obiettivo è analizzare chi viene colpito e con quale frequenza, incrociando le tipologie di bersaglio (\texttt{targtype1\_txt}) con le aree geografiche (\texttt{region\_txt}).

Al fine di valutare il "Rischio di essere colpiti", la metrica utilizzata non è più la somma dei decessi, ma il \textbf{Conteggio degli Eventi} (Frequenza). Questo permette di identificare quali settori sono strutturalmente più sotto tiro, indipendentemente dall'esito letale del singolo attacco. La dashboard si compone di due strumenti complementari: un'analisi dinamica temporale (a sinistra) e una mappatura statica del rischio (a destra).

\textcolor{red}{\textbf{foto}}


\begin{enumerate}
    \item \textbf{Evoluzione Temporale e Previsione (Stacked Area Chart):} La parte sinistra della dashboard ospita un grafico ad aree impilate che mostra la composizione dei bersagli dal 1970 al 2017. I settori analizzati sono i cinque principali: \textit{Business, Government (General), Military, Police,} e \textit{Private Citizens \& Property}.
    \begin{itemize}
        \item \textbf{Analisi Storica Visiva:} Il grafico evidenzia come la distribuzione dei bersagli non sia costante. Si nota una crescita significativa a cavallo tra gli anni '80 e '90, seguita da una flessione. Tuttavia, il dato più allarmante è l'esplosione volumetrica post-2010 (anche in questo grafico è visibile il "buco" fisiologico del 1993).
        \item \textbf{Strategic Forecasting (Previsione a 5 anni):} Il valore aggiunto di questo grafico è la proiezione futura (2017--2021), rappresentata dall'area più chiara a destra. Questa estensione supporta i processi di \textit{Strategic Planning} a medio termine dei governi. L'intervallo di confidenza associato si allarga progressivamente, indicando matematicamente l'aumento dell'incertezza man mano che ci si allontana dai dati storici.
    \end{itemize}

    \item \textbf{Matrice di Concentrazione degli Attacchi (Heatmap):} La parte destra della dashboard risponde alla necessità di localizzare geograficamente la minaccia tramite una \textit{Highlight Table} (Tabella Termica). Incrociando le Regioni (sulle righe) con i Target (sulle colonne), l'intensità del colore mappa la densità storica degli attacchi, fungendo da vera e propria mappa di rischio statica.
    \begin{itemize}
        \item \textbf{Identificazione delle "Zone Rosse":} Analizzando i dati crudi riportati nella matrice, emergono chiaramente le combinazioni a massimo rischio globale. Le celle dal colore più intenso si trovano in \textit{Middle East \& North Africa}, dove gli attacchi ai \textit{Private Citizens} raggiungono la cifra record di 15.257 eventi storici, seguiti dagli attacchi alla \textit{Police} (6.893). Anche il \textit{South Asia} mostra una densità critica, con 10.491 attacchi ai privati cittadini e 8.471 ai militari.
        \item Al contrario, regioni come \textit{Australasia \& Oceania} mostrano celle chiarissime (poche decine di attacchi), confermando un profilo di rischio quasi nullo.
    \end{itemize}

    \item \textbf{Interattività e Risposta alle Business Questions:} Il collante dell'intera dashboard è il \textbf{Filtro Dinamico per Regione}, posizionato in alto a destra. Selezionando un'area specifica (es. \textit{Western Europe}), sia l'\textit{Area Chart} che la previsione si ricalcolano immediatamente. Questo permette agli analisti di rispondere a domande di business mirate, come ad esempio verificare se in Europa si stia assistendo a uno spostamento dei bersagli dalle istituzioni (\textit{Government}) verso i mercati civili (\textit{Business} o \textit{Private Citizens}).
\end{enumerate}

\subsubsection{Validazione Statistica e Comportamento del Modello Predittivo}

\paragraph{Motore di Forecasting e Orizzonte Temporale}
Per la proiezione a 5 anni (2017--2021) dei volumi di attacco sui diversi bersagli, il motore analitico ha utilizzato il Livellamento Esponenziale sui dati storici dal 1970 al 2016. Coerentemente con la granularità annuale del dataset, il software non ha forzato alcuna componente di stagionalità (Modello Stagione: \textit{Nessuna}).

\paragraph{Il Cambiamento di Paradigma (Soft Target vs Hard Target)}
L'analisi dei coefficienti del modello rivela una netta spaccatura matematica tra gli obiettivi civili e quelli istituzionali, confermando l'ipotesi visiva di un cambio di strategia del terrorismo moderno:



\begin{enumerate}
    \item \textbf{Istituzioni, Forze dell'Ordine e Business (Hard/Economic Targets):} Per categorie come \texttt{Police}, \texttt{Military}, \texttt{Government} e \texttt{Business}, il coefficiente Beta ($\beta$, che regola la pendenza del trend) è pari a $0,000$. Di conseguenza, il modello prevede una stabilizzazione della minaccia: la stima di crescita per il quinquennio successivo è pari a $0$ rispetto al valore iniziale.
    \item \textbf{Civili e Proprietà Private (Soft Targets):} Per la categoria \texttt{Private Citizens \& Property}, il coefficiente $\beta$ è salito a $0,297$, portando il modello a calcolare un incremento stimato impressionante di $+2.084$ attacchi nel quinquennio. Questo dato certifica matematicamente lo spostamento della violenza verso bersagli non difesi.
\end{enumerate}

\textcolor{red}{\textbf{Foto dati confronto}}



\paragraph{Analisi delle Metriche di Qualità (Il significato del rating "Scarso")}
Come già osservato nell'analisi della letalità (Dashboard 2), Tableau classifica la qualità complessiva di queste previsioni come \textit{"Scarsa"}. Questa valutazione è il riflesso dell'alta volatilità geopolitica. Esaminando le metriche specifiche:

\begin{itemize}
    \item \textbf{MAPE :} Si attesta su valori che vanno dal $27,7\%$ per il settore \textit{Business} fino a un picco del $50,1\%$ per i \textit{Private Citizens}. L'errore così alto sui civili è dovuto all'impennata anomala ed esponenziale degli attacchi nell'ultimo decennio, che rende i dati storici del passato meno affidabili per prevedere l'entità esatta della crescita futura.
    \item \textbf{MASE :} I valori oscillano tra $1,01$ e $1,11$. Essendo leggermente superiori a $1$, indicano che, a causa delle brusche oscillazioni (cicli di pace seguiti da improvvise ondate di attentati), il modello di smussamento esponenziale fatica a performare significativamente meglio di una semplice "previsione Naive" (che replicherebbe linearmente l'ultimo dato noto).
\end{itemize}

\paragraph{Implicazioni Strategiche}
L'alta varianza (testimoniata dall'ampio intervallo di confidenza, es. $\pm 780$ attacchi sui civili) e la netta distinzione dei trend modellati suggeriscono un'importante indicazione di \textit{Intelligence}: mentre le misure di sicurezza fisiche a protezione di basi militari e palazzi governativi sembrano aver "congelato" la crescita degli attacchi contro di essi (trend piatto), i terroristi hanno compensato rivolgendosi verso gli spazi pubblici. La previsione indica ai decisori politici che la vera priorità strategica dei prossimi anni sarà la messa in sicurezza dei \textit{Soft Target} urbani, dove l'incertezza e il volume atteso sono massimi.

\textcolor{red}{\textbf{Foto dati}}


\subsection{Dashboard 4: Analisi degli Esiti dell'Attacco (Successo vs Fallimento)}

\subsubsection*{Obiettivo dell'Analisi}
Mentre le dashboard precedenti hanno esaminato la frequenza, la letalità e la natura dei bersagli, questa visualizzazione sposta il focus sull'efficacia delle misure di sicurezza e dell'intelligence antiterrorismo. L'analisi si basa sulla variabile binaria dell'esito dell'attacco: \textbf{Successo} (l'attacco è stato portato a termine come pianificato) o \textbf{Fallimento} (l'ordigno è stato disinnescato, il terrorista intercettato o l'azione è fallita per cause tecniche).

L'obiettivo strategico è comprendere se le capacità difensive globali stiano migliorando nel tempo e se la natura del bersaglio influisca sulla probabilità di riuscita dell'attentato. La dashboard è composta da due visualizzazioni temporali sincronizzate e governate da un filtro globale per tipologia di bersaglio.

\textcolor{red}{\textbf{Foto}}


\begin{enumerate}
    \item \textbf{Analisi Temporale dell'Efficacia (Composizione e Trend):} La porzione principale della dashboard risponde alla domanda su come variano i risultati storicamente.
    \begin{itemize}
        \item \textbf{100\% Stacked Bar Chart:} Il grafico a barre in pila normalizzato al 100\% mostra visivamente la proporzione annua tra attacchi riusciti (porzione Arancione) e attacchi falliti (porzione Viola). A livello globale, emerge un dato strutturale: il terrorismo ha un tasso di successo fisiologicamente altissimo. Questo è tipico delle tattiche asimmetriche, in cui l'attaccante ha il vantaggio della sorpresa.
        \item \textbf{Trend Line dell'Efficacia (Tasso di Fallimento):} Sotto l'istogramma, un grafico a linee isola e traccia unicamente la percentuale dei fallimenti nel corso dei decenni.
        \begin{itemize}
            \item \textit{Analisi Storica:} Si nota un picco anomalo di fallimenti nei primi anni '70 (intorno al 1972, prossimo al 20\%).
            \item \textit{Il Trend Recente:} Il dato di business più rilevante si osserva nella parte finale della curva (dal 2010 al 2017). La linea del tasso di fallimento mostra una crescita rapida e costante, passando da valori sotto il 10\% a picchi che superano il 15\%--20\%. Questo trend in salita è un indicatore positivo: certifica che l'implementazione di moderne tecnologie di prevenzione, la sorveglianza digitale e il miglioramento delle operazioni di \textit{intelligence} stanno portando a sventare un numero sempre maggiore di attentati.
        \end{itemize}
    \end{itemize}

    \item \textbf{Correlazione tra Bersaglio ed Esito (Analisi "Hard vs Soft Target"):}
    Per rispondere al quesito se sia più difficile colpire una base militare rispetto a una piazza pubblica, la dashboard è stata dotata di un \textbf{Filtro Dinamico per Target} (\texttt{targtype1\_txt}). Questo strumento interattivo permette di condurre un'analisi differenziale immediata:



    \begin{itemize}
        \item \textbf{Hard Targets (es. Military / Police):} Selezionando bersagli istituzionali, dotati di protocolli di sicurezza attivi, perimetri fortificati e personale armato, la linea del "Tasso di Fallimento" tende a posizionarsi su valori percentuali mediamente più alti. Le difese passive e attive fungono da deterrente o riescono a neutralizzare la minaccia durante la fase di esecuzione.
        \item \textbf{Soft Targets (es. Private Citizens \& Property):} Cambiando il filtro su obiettivi civili (luoghi pubblici, piazze, mercati), si osserva un restringimento della barra viola e un crollo della linea dei fallimenti. Privi di difese strutturali, questi bersagli garantiscono ai terroristi una probabilità di successo dell'azione quasi assoluta.
    \end{itemize}
\end{enumerate}

\subsubsection*{Conclusioni Strategiche}
Questa dashboard chiude il cerchio dell'analisi, trasformando una variabile binaria in un misuratore di \textit{performance} (KPI) per la sicurezza globale. La visualizzazione dimostra che, sebbene spostare l'attenzione sui \textit{Soft Target} garantisca ai terroristi tassi di successo altissimi, gli sforzi di prevenzione sistemica degli ultimi anni stanno invertendo la rotta, registrando i tassi di fallimento dell'attacco più alti degli ultimi 40 anni.



\subsubsection{Caso Studio Comparativo: \textit{Military} vs \textit{Airports \& Aircraft}}
Per rispondere al quesito specifico se sia più difficile colpire una base militare rispetto a un aeroporto, sono stati isolati i dati di queste due categorie tramite il filtro dinamico, analizzando le rispettive linee di tendenza dei fallimenti. Il confronto ha rivelato un dato statistico di grande rilevanza strategica.



\begin{itemize}
    \item \textbf{Analisi del Target \textit{Military} (Basi e Forze Armate):} Filtrando per obiettivi militari, il grafico a barre mostra che il tasso di successo degli attacchi rimane storicamente molto alto. Osservando la \textit{Trend Line} del tasso di fallimento, notiamo che per decenni (dal 1970 al 2014) la percentuale di attacchi sventati ha oscillato in una fascia relativamente bassa, tra il 5\% e il 10\%. Solo nel biennio finale (2016--2017) si registra un'impennata che supera il 20\%.

    \textit{Interpretazione:} Sebbene le basi militari siano \textit{Hard Targets} (obiettivi fortificati), le truppe operano spesso in zone di conflitto attivo o in scenari asimmetrici, dove sono costantemente esposte ad agguati, ordigni improvvisati, rendendo la prevenzione assoluta molto complessa.

    \item \textbf{Analisi del Target \textit{Airports \& Aircraft} (Aviazione Civile e Infrastrutture):} Cambiando il filtro sul settore dell'aviazione, lo scenario cambia drasticamente. La prima evidenza tecnica è il cambio di scala dell'asse Y del grafico del Trend di Efficacia: mentre per i militari il tetto massimo era intorno al 20\%, qui l'asse arriva all'80\%. Storicamente, l'aviazione mostra tassi di fallimento molto più alti (spesso tra il 15\% e il 30\% già negli anni passati), con un picco anomalo e verticale nell'ultimo anno registrato (2017), dove la percentuale di attacchi falliti o sventati schizza quasi all'80\%. Il grafico a barre soprastante conferma visivamente questo dato, mostrando ampie porzioni (fallimenti) che dominano in diverse annate.
\end{itemize}

Questa marcata discrepanza deriva dalla natura delle difese:
\begin{enumerate}
    \item \textbf{Standardizzazione Globale:} Il settore aeroportuale è regolato da protocolli di sicurezza internazionali standardizzati (\textit{metal detector}, \textit{scanner} a raggi X, controlli incrociati sui passeggeri, \textit{intelligence} preventiva) che bloccano la minaccia prima che entri nella fase esecutiva.
    \item \textbf{Vulnerabilità Tattica:} Un aeroporto è un ambiente chiuso e controllato, dove l'attaccante deve superare filtri successivi. Un convoglio militare, al contrario, si trova spesso in campo aperto, dove l'attaccante può sfruttare il fattore sorpresa o la forza bruta.
\end{enumerate}

\textcolor{red}{\textbf{Foto esempio }}
