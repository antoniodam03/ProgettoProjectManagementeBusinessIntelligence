\phantomsection
\addcontentsline{toc}{chapter}{Conclusioni}
\chapter*{Conclusioni}
\markboth{Conclusioni}{}

Il presente lavoro ha dimostrato come l'applicazione di metodologie avanzate di \textit{Business Intelligence} e \textit{Data Analytics} possa trasformare vasti volumi di dati grezzi, relativi a decenni di eventi di violenza politica e terrorismo, in un ecosistema informativo strategico (\textit{Actionable Insights}). L'obiettivo primario di supportare i complessi processi decisionali nei settori della Difesa e della Sicurezza è stato raggiunto attraverso lo sviluppo di un'architettura analitica completa, scalabile e multidimensionale.

Partendo da una rigorosa fase di \textit{Data Preparation} (ETL) implementata in Python, è stato possibile bonificare e strutturare il \textit{Global Terrorism Database} (GTD), risolvendo problematiche di valori nulli e \textit{type casting} per garantire un'elevata \textit{Data Quality}. Il vero valore ingegneristico del progetto, tuttavia, risiede nell'approccio metodologico adottato per l'analisi esplorativa e visiva. Invece di limitarsi alla mera visualizzazione, il dominio dei dati è stato indagato attraverso una ``struttura a spirale'', impiegando tre dei principali leader di mercato nel settore BI per rispondere a specifiche interrogazioni di business, ciascuno secondo la propria vocazione:

\begin{itemize}
    \item \textbf{L'Analisi Descrittiva (Qlik):} Sfruttando il motore associativo, è stato possibile mappare storicamente e geograficamente il fenomeno, quantificando l'impatto economico e le frequenze delle diverse tattiche adottate, offrendo a \textit{Risk Manager} e decisori una \textit{Data Discovery} rapida e intuitiva.
    
    \item \textbf{L'Analisi Predittiva (Tableau):} Attraverso modelli statistici come lo smussamento esponenziale (\textit{Exponential Smoothing}) e metriche di validazione, si è passati dall'osservazione del passato alla previsione dei trend futuri, proiettando i potenziali danni economici e i tassi di fallimento degli attacchi.
    
    \item \textbf{L'Analisi Diagnostica (Power BI):} L'integrazione di algoritmi di \textit{Machine Learning} (come alberi di scomposizione e \textit{Key Influencers}) ha permesso di scavare nelle cause radice degli eventi, identificando le correlazioni nascoste tra l'uso di specifici armamenti, la probabilità di causare massacri e la cosiddetta ``anatomia del fallimento''.
\end{itemize}

In conclusione, le \textit{dashboard} realizzate non si limitano a fornire una fotografia statica della minaccia terroristica globale, ma si configurano come un vero e proprio Strumento di Supporto alle Decisioni (DSS). Permettono agli analisti di valutare i rischi in modo oggettivo, ottimizzare l'allocazione delle risorse difensive e anticipare l'evoluzione delle tattiche avversarie.